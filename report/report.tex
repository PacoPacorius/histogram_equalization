\documentclass{article} 
\usepackage{polyglossia} 
\usepackage{amsmath}
\usepackage{fontspec} 
\usepackage{lipsum} 
\usepackage[margin=1in]{geometry}
\usepackage{graphicx} 
\usepackage{caption} 
\usepackage{subcaption}
\usepackage{hyperref} 
\usepackage{verbatim}
\hypersetup{% 
    colorlinks=true, linkcolor=blue, filecolor=magenta,      
    urlcolor=cyan, 
    pdfinfo = {%
        Title = {ΨΕΕ 1η εργασία}
        Author = {Χρήστος Μάριος Περδίκης},
        Producer = XeLaTeX
    } 
}


\setmainfont{FreeSerif}


\title{Ψηφιακή Επεξεργασία Εικόνας \\ Παραλλαγές αλγορίθμων εξισορρόπησης και 
αντιστοίχισης ιστογράμματος}
\date{Εαρινό Εξάμηνο 2024-2025}
\author{Χρήστος-Μάριος Περδίκης 10075 cperdikis@ece.auth.gr}
 
\begin{document}
\maketitle

Αυτή είναι η αναφορά για την 1η εργασία του μαθήματος Ψηφιακής Επεξεργασίας 
Εικόνας. Ο σκοπός της εργασίας είναι η υλοποίηση μεθόδων εξισορρόπησης και 
αντιστοίχησης ιστογράμματος εικόνας σε python. Ακολουθεί η επεξήγηση των 
παραδοτέων συναρτήσεων.

\section{Συναρτήσεις στο αρχείο \texttt{hist\_utils.py}}
\subsection{\texttt{calculate\_hist\_of\_img()}}
Αρχικοποιείται το dictionary $hist$ με μήκος 256, ώστε να καλύψει όλες τις δυνατές 
τιμές μιας 8-bit εικόνας. Για κάθε pixel της εικόνας που έχει τιμή $f_i$, 
η αντίστοιχη τιμή του $hist$ αυξάνεται κατά ένα, δηλαδή $hist[f_i] = hist[f_i] + 1$.
Η συνάρτηση επιστρέφει το $hist$.

\subsection{\texttt{apply\_hist\_modification\_transform()}}
Αρχικοποιείται η νέα εικόνα $modified\_image$. Το κάθε pixel $(x, y)$ της λαμβάνει
τιμή $g_i$ που προέκυψε από την αντιστοιχία τιμών $g_i \leftrightarrow f_i$ μέσω από το
Dict $modification\_transform$, όπου $f_i$ η τιμή του αντίστοιχου pixel $(x, y)$
στην εικόνα εισόδου $img\_array$. Η συνάρτηση επιστρέφει την εικόνα $modified\_image$.

\section{Συναρτήσεις στο αρχείο \texttt{hist\_modif.py}}
\subsection{\texttt{perform\_hist\_modification()}}
Ανάλογα με την τιμή της μεταβλητής $mode$, `greedy', `non-greedy' και
`post-disturbance', εκτελείται η αντίστοιχη μέθοδος αντιστοίχησης ιστογράμματος
μεταξύ του ιστογράμματος της εικόνας εισόδου $img\_array$ και του ιστο\-γράμματος
αναφοράς $hist\_ref$. Το πρώτο υπολογίζεται 
από τη συνάρτηση \verb|calculate_hist_of_img()| και μετά
καλείται μία από τις βοηθητικές συναρτήσεις \verb|greedy_histogram_matching()|, 
\verb|non_greedy_histogram_|\-\verb|matching()| και 
\verb|post_disturbance_histogram_matching()| ανάλογα με την τιμή της 
$mode$. Η αναλυτική επεξήγηση των συναρτήσεων αυτών βρίσκεται στις 
υποενότητες~\ref{helpfunc-greedy},~\ref{helpfunc-nongreedy} και~\ref{helpfunc-post}.
Για την ώρα, καλούνται με ορίσματα τα δύο ιστογράμματα, ένα της εικόνας εισόδου και ένα
αναφοράς, και επιστρέφουν την αντιστοίχηση μεταξύ των δύο ιστογραμμάτων σε 
μορφή Dict με όνομα $mod\_transform$. Τέλος, 
καλείται η συνάρτηση \verb|apply|\-\verb|_hist_modification()| με ορίσματα την $img\_array$
και το $mod\_transform$, η έξοδός της είναι και η έξοδος της 
\verb|perform_hist_modification()|.

\subsection{\texttt{perform\_hist\_eq()}}
Σε αυτή τη συνάρτηση καλείται η \verb|perform_hist_modification()| με ορίσματα 
την εικόνα εισόδου $img\_array$ και ένα ομοιόμορφο ιστόγραμμα. Το ομοιόμορφο 
ιστόγραμμα κατασκευάζεται ως εξής. Αρχικά υπολογίζεται το $img\_size$ πλήθος όλων των pixels 
της $img\_array$. Έπειτα αρχικοποιείται ένα Dict $uniform\_hist$ με 256
θέσεις, τα κλειδιά του είναι ακέραιες τιμές στο διάστημα $\left[0, 255\right]$ και
οι τιμές του έχουν όλες την ίδια τιμή $\frac{img\_size}{256}$. Έτσι το 
$uniform\_hist$ είναι το κατάλληλο ομοιογενές flat ιστόγραμμα. Η έξοδος της 
\verb|perform_hist_modification()| είναι και η έξοδος της \verb|perform_hist_eq()|.

\subsection{\texttt{perform\_hist\_matching()}}
Χρειάζεται μόνο να κλιθεί η \verb|perform_hist_modification()| με ορίσματα την 
εικόνα εισόδου $img\_array$ και το ιστόγραμμα της εικόνας αναφοράς. Αυτό
υπολογίζεται με κλίση της \verb|calculate_hist_of_img()|.

\section{Αρχείο επίδειξης \texttt{demo.py}}
Αφού φορτωθούν οι δύο εικόνες ``input\_img.py'' και ``ref\_img.py'' με τη χρήση 
της opencv καλείται η βοηθητική συνάρτηση \verb|display_images_and_histograms()|.
Η επεξήγησή της βρίσκεται στην υποενότητα~\ref{helpfunc-display}. Με τη χρήση 
αυτής της συνάρτησης προβάλλονται οι δύο αρχικές εικόνες. Έπειτα καλείται οι
\verb|perform_hist_eq()| με όρισμα την ``input\_img.py'' τρεις φορές, μία
για κάθε mode (`greedy', `non-greedy' και `post-disturbance') και ξανακαλείται η
συνάρτηση~\ref{helpfunc-display} για να προάλλει την αρχική εικόνα ``input\_img.py''
και τις τρεις εικόνες που προκύπτουν από τις τρεις μεθόδους εξισορρόπησης 
ιστογράμματος. Όμοια διαδικασία γίνεται και για την αντιστοίχηση ιστογράμματος, 
καλείται τρεις φορές η \verb|perform_hist_matching()| για όλα τα modes και 
προβά\-λλονται η αρχική εικόνα, η εικόνα αναφοράς και οι τρεις εικόνες-αποτελέσματα 
των αντιστοιχίσεων ιστογράμ\-ματος.

\section{Βοηθητικές συναρτήσεις}
\subsection{\texttt{greedy\_histogram\_matching()}}\label{helpfunc-greedy}
\hrule
\vspace{1em}
\begin{verbatim}
greedy_histogram_matching(input_hist: Dict, ref_hist: Dict) -> Dict:
\end{verbatim}
\hrule
\vspace{1em}


\subsection{\texttt{non\_greedy\_histogram\_matching()}}\label{helpfunc-nongreedy}
\hrule
\vspace{1em}
\begin{verbatim}
non_greedy_histogram_matching(input_hist: Dict, ref_hist: Dict) -> Dict:
\end{verbatim}
\hrule
\vspace{1em}

\subsection{\texttt{post\_disturbance\_histogram\_matching()}}\label{helpfunc-post}
\hrule
\vspace{1em}
\begin{verbatim}
post_disturbance_histogram_matching(input_hist: Dict, ref_hist: Dict) -> Dict:
\end{verbatim}
\hrule
\vspace{1em}

\subsection{\texttt{display\_images\_and\_histograms()}}\label{helpfunc-display}
\section{Αποτελέσματα}
Μετά την εξισορρόπηση ιστογράμματος μπορούμε να παρατηρήσουμε ότι οι περιοχές 
του ιστογράμματος που προηγουμένως υπήρχαν πολλά μαζεμένα pixel απλώθηκαν σε ένα 
μεγαλύτερο εύρος τιμών. Βέβαια στις περιοχές της εικόνας που επηρεάστηκαν από
την εξισορρόπηση ιστογράμματος υπάρχον μη ομαλές μεταβάσεις από ένα επίπεδο
φωτεινότητας σε ένα άλλο. Αυτό φαίνεται άσχημο στο μάτι. Δεν φαίνεται να υπάρχει 
καμιά διακριτή διαφορά μεταξύ των διαφορετικών μεθόδων `greedy', `non-greedy' και
`post-disturbance'.

Μετά την αντιστοίχηση ιστογράμματος βλέπουμε πράγματι την αρχική εικόνα μας να 
έχει φωτεινότητες κοντινότερες σε αυτές της εικόνας αναφοράς. Όπως την εικόνα
αναφοράς, το αποτέλεσμα φαίνεται να είναι ποι σκοτεινό γενικά. Υπάρχει ένα 
σημείο πάνω στη γάτα το οποίο είναι πολύ φωτεινό, σε αυτό φαίνεται διαφορά 
ανάμεσα των τριών μεθόδων. Η `post-disturbance' μέθοδος είναι πιο φωτεινή σε
εκείνο το σημείο. Στα ιστόγραμματα φαίνεται επίσης ότι η `post-disturbance'
μέθοδος έχει διαφορά από τις άλλες δύο στις χαμηλές φωτεινότητες.

\end{document}
