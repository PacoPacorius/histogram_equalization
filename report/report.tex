\documentclass{article} 
\usepackage{polyglossia} 
\usepackage{amsmath}
\usepackage{fontspec} 
\usepackage{lipsum} 
\usepackage[margin=1in]{geometry}
\usepackage{graphicx} 
\usepackage{caption} 
\usepackage{subcaption}
\usepackage{hyperref} 
\usepackage{verbatim}
\hypersetup{% 
    colorlinks=true, linkcolor=blue, filecolor=magenta,      
    urlcolor=cyan, 
    pdfinfo = {%
        Title = {ΨΕΕ 1η εργασία}
        Author = {Χρήστος Μάριος Περδίκης},
        Producer = XeLaTeX
    } 
}


\setmainfont{FreeSerif}


\title{Ψηφιακή Επεξεργασία Εικόνας \\ Παραλλαγές αλγορίθμων εξισορρόπησης και 
αντιστοίχισης ιστογράμματος}
\date{Εαρινό Εξάμηνο 2024-2025}
\author{Χρήστος-Μάριος Περδίκης 10075 cperdikis@ece.auth.gr}
 
\begin{document}
\maketitle

Αυτή είναι η αναφορά για την 1η εργασία του μαθήματος Ψηφιακής Επεξεργασίας 
Εικόνας. Ο σκοπός της εργασίας είναι η υλοποίηση μεθόδων εξισορρόπησης και 
αντιστοίχησης ιστογράμματος εικόνας σε python. Ακολουθεί η επεξήγηση των 
παραδοτέων συναρτήσεων.

\section{Συναρτήσεις στο αρχείο \texttt{hist\_utils.py}}
\subsection{\texttt{calculate\_hist\_of\_img(img\_array: np.ndarray, return\_normalized: bool)}}
Αρχικοποιείται το dictionary $hist$ με μήκος 256, ώστε να καλύψει όλες τις δυνατές 
τιμές μιας 8-bit εικόνας. Για κάθε pixel της εικόνας που έχει τιμή $f_i$, 
η αντίστοιχη τιμή του $hist$ αυξάνεται κατά ένα, δηλαδή $hist[f_i] = hist[f_i] + 1$.
Η συνάρτηση επιστρέφει το $hist$.

\subsection{\texttt{apply\_hist\_modification\_transform(img\_array: np.ndarray, \\ modification\_transform: Dict)}}
Αρχικοποιείται η νέα εικόνα $modified\_image$. Το κάθε pixel $(x, y)$ της λαμβάνει
τιμή $g_i$ που προέκυψε από την αντιστοιχία τιμών $g_i \leftrightarrow f_i$ μέσω από το
Dict $modification\_transform$, όπου $f_i$ η τιμή του αντίστοιχου pixel $(x, y)$
στην εικόνα εισόδου $img\_array$. Η συνάρτηση επιστρέφει την εικόνα $modified\_image$.

\section{Συναρτήσεις στο αρχείο \texttt{hist\_modif.py}}
\subsection{\texttt{perform\_hist\_modification(img\_array: np.ndarray, hist\_ref: Dict, mode: str)}}
Ανάλογα με την τιμή του ορίσματος $mode$, `greedy', `non-greedy' και
`post-disturbance', εκτελείται η αντίστοιχη μέθοδος αντιστοίχησης ιστογράμματος
μεταξύ του ιστογράμματος της εικόνας εισόδου $img\_array$ και του ιστο\-γράμματος
αναφοράς $hist\_ref$. Το πρώτο υπολογίζεται 
από τη συνάρτηση \verb|calculate_hist_of_img()| και μετά
καλείται μία από τις βοηθητικές συναρτήσεις \verb|greedy_histogram_matching()|, 
\verb|non_greedy_histogram_|\-\verb|matching()| και 
\verb|post_disturbance_histogram_matching()| ανάλογα με την τιμή της 
$mode$. Η αναλυτική επεξήγηση των συναρτήσεων αυτών βρίσκεται στις 
υποενότητες~\ref{helpfunc-greedy},~\ref{helpfunc-nongreedy} και~\ref{helpfunc-post}.
Για την ώρα, καλούνται με ορίσματα τα δύο ιστογράμματα, ένα της εικόνας εισόδου και ένα
αναφοράς, και επιστρέφουν την αντιστοίχηση μεταξύ των δύο ιστογραμμάτων σε 
μορφή Dict με όνομα $mod\_transform$. Τέλος, 
καλείται η συνάρτηση \verb|apply|\-\verb|_hist_modification()| με ορίσματα την $img\_array$
και το $mod\_transform$, η έξοδός της είναι και η έξοδος της 
\verb|perform_hist_modification()|.

\subsection{\texttt{perform\_hist\_eq(img\_array: np.ndarray, mode: str)}}
Σε αυτή τη συνάρτηση καλείται η \verb|perform_hist_modification()| με ορίσματα 
την εικόνα εισόδου $img\_array$ και ένα ομοιόμορφο ιστόγραμμα. Το ομοιόμορφο 
ιστόγραμμα κατασκευάζεται ως εξής. Αρχικά υπολογίζεται το $img\_size$ πλήθος όλων των pixels 
της $img\_array$. Έπειτα αρχικοποιείται ένα Dict $uniform\_hist$ με 256
θέσεις, τα κλειδιά του είναι ακέραιες τιμές στο διάστημα $\left[0, 255\right]$ και
οι τιμές του έχουν όλες την ίδια τιμή $\frac{img\_size}{256}$. Έτσι το 
$uniform\_hist$ είναι το κατάλληλο ομοιογενές flat ιστόγραμμα. Η έξοδος της 
\verb|perform_hist_modification()| είναι και η έξοδος της \verb|perform_hist_eq()|.

\subsection{\texttt{perform\_hist\_matching(img\_array: np.ndarray, img\_array\_ref: np.ndarray, mode: str)}}
Χρειάζεται μόνο να κλιθεί η \verb|perform_hist_modification()| με ορίσματα την 
εικόνα εισόδου $img\_array$ και το ιστόγραμμα της εικόνας αναφοράς. Το ιστόγραμμα
αυτό υπολογίζεται με κλίση της \verb|calculate_hist_of_img()|.

\section{Αρχείο επίδειξης \texttt{demo.py}}
Αφού φορτωθούν οι δύο εικόνες ``input\_img.py'' και ``ref\_img.py'' με τη χρήση 
της opencv καλείται η βοηθητική συνάρτηση \verb|display_images_and_histograms()|.
Η επεξήγησή της βρίσκεται στην υποενότητα~\ref{helpfunc-display}. Με τη χρήση 
αυτής της συνάρτησης προβάλλονται οι δύο αρχικές εικόνες. Έπειτα καλείται οι
\verb|perform_hist_eq()| με όρισμα την ``input\_img.py'' τρεις φορές, μία
για κάθε mode (`greedy', `non-greedy' και `post-disturbance') και ξανακαλείται η
συνάρτηση~\ref{helpfunc-display} για να προβάλλει την αρχική εικόνα ``input\_img.py''
και τις τρεις εικόνες που προκύπτουν από τις τρεις μεθόδους εξισορρόπησης 
ιστογράμματος. Όμοια διαδικασία γίνεται και για την αντιστοίχηση ιστογράμματος, 
καλείται τρεις φορές η \verb|perform_hist_matching()| για όλα τα modes και 
προβά\-λλονται η αρχική εικόνα, η εικόνα αναφοράς και οι τρεις εικόνες-αποτελέσματα 
των αντιστοιχίσεων ιστογράμ\-ματος.

\section{Βοηθητικές συναρτήσεις}
Οι συναρτήσεις~\ref{helpfunc-greedy}, \ref{helpfunc-nongreedy} και~\ref{helpfunc-post} βρίσκονται στο αρχείο ``hist\_modif.py'' και 
η~\ref{helpfunc-display} στο αρχείο ``demo.py''

\subsection{\texttt{greedy\_histogram\_matching(input\_hist: Dict, ref\_hist: Dict)}}\label{helpfunc-greedy}
Στην υλοποίηση της μεθόδου greedy histogram matching δημιουργείται η λίστα 
$input\_intensities$ η οποία αρχι\-κοποι\-είται με
κάθε τιμή φωτεινότητας $count$ φορές, όπου $count$ η 
συχνότητα εμφάνισης της τιμής αυτής στην αρχική εικόνα. Αυτή η 
πληροφορία υπάρχει στο Dict $input\_hist$. Η προσπέλαση των τιμών 
φωτεινότητας γίνεται με αύξουσα σειρά.
Η ίδια διαδικασία επαναλαμβάνεται για το ιστόγραμμα αναφοράς
και τη λίστα $ref\_intensities$.

Στην περίπτωση αντιστοίχισης ιστογράμματος, οι δύο λίστες 
μπορεί να μην έχουν το ίδιο μήκος. Αν η λίστα $input\_intensities$ είναι μεγαλύτερη, 
υπολογίζεται η αναλογία $ratio = \frac{len(input\_intensities)}{len(ref\_intensities)}$,
Η $ref\_intensities$ επεκτείνεται μέχρι να έχει μήκος 
$len(input\_intensities) / ratio$. Στις νέες θέσεις της επαναλαμβάνονται
ήδη υπάρχουσες τιμές της με τέτοιο τρόπο έτσι ώστε π.χ. τα πρώτα δύο pixel
της εικόνας εισόδου να αντιστοιχί\-ζονται με το πρώτο
επίπεδο φωτεινότητας της εικόνας αναφοράς, αν $ratio = 2$. 
Δηλαδή η αντιστοίχηση γίνεται έτσι ώστε,
αν $i$ αναφέρονται σε $input\_intensities$ και $j$ σε 
$ref\_intensities$, $ i / ratio = j$.
αν η λίστα της εικόνας αναφοράς είναι μεγαλύτερη, 
πάλι υπολογίζεται το $ratio = \frac{len(input\_intensities)}{len(ref\_intensities)}$
και μειώνεται το μήκος του $ref\_intensities$ έτσι ώστε,
αν $i$ αναφέρεται σε θέση της λίστας $input\_intensities$ και $j$ σε 
θέση της $ref\_intensities$ πριν την επέκταση, η αντιστοίχηση να είναι $i * ratio  = j$.
Η έξοδος της συνάρτησης είναι το Dict $transform$ που ταιριάζει
την τιμή $input\_intensities\left[i\right]$ με την τιμή $ref\_intensities\left[i\right]$.

\subsection{\texttt{non\_greedy\_histogram\_matching(input\_hist: Dict, ref\_hist: Dict)}}\label{helpfunc-nongreedy}

Στη non-greedy μέθοδο, πρώτα καλείται η συνάρτηση~\ref{helpfunc-greedy} και μετά γίνεται έλεγχος
αν οποιαδήποτε τιμή της εξόδου της $transform$ είναι μικρότερη από οποιαδήποτε προηγούμενη τιμή
της. Αν είναι, τότε η μικρότερη τιμή εξισώνεται με τη μεγαλύτερη. Το προκύπτον ιστόγραμμα 
δεν είναι τόσο ακριβές όσο με την απλή συνάρτηση~\ref{helpfunc-greedy}, αλλά διατηρούνται
καλύτερα οι σχέσεις μεταξύ των διαφορών φωτεινότητας στην τελική εικόνα.

\subsection{\texttt{post\_disturbance\_histogram\_matching(input\_hist: Dict, ref\_hist: Dict)}}\label{helpfunc-post}

Στη post-disturbance μέθοδο πρώτα γίνεται εξισορρόπηση ιστογράμματος στην εικόνα 
εισόδου (εισάγεται ένα disturbance) και μετά το εξισορροπημένο ιστόγραμμα 
γίνεται αντιστοίχηση με το ιστόγραμμα αναφοράς. Και οι δύο λειτουργίες γίνονται 
με τις κατάλληλες κλίσεις στην συνάρτηση~\ref{helpfunc-greedy}. Τα δύο transforms 
που προκύπτουν, $equalized\_transform$ για την εξισορρόπηση ιστογράματος και 
$uniform\_to\_ref\_transform$ για την αντιστοί\-χιση συνδυάζονται και η
έξοδος της συνάρτησης είναι το Dict $combined\_transform$.

\subsection{\texttt{display\_images\_and\_histograms()}}\label{helpfunc-display}

Σε αυτή τη συνάρτη γίνεται χρήση του module matplotlib για να 
προβληθούν τα αποτελέσματα της επεξεργασίας των ιστογραμμάτων. 

\section{Αποτελέσματα}
Μετά την εξισορρόπηση ιστογράμματος μπορούμε να παρατηρήσουμε ότι οι περιοχές 
του ιστογράμματος που προηγουμένως υπήρχαν πολλά μαζεμένα pixel απλώθηκαν σε ένα 
μεγαλύτερο εύρος τιμών φωτεινότητας. Βέβαια στις περιοχές της εικόνας που επηρεάστηκαν από
την εξισορρόπηση ιστογράμματος υπάρχουν μη ομαλές μεταβά\-σεις από ένα επίπεδο
φωτεινότητας σε άλλο. Αυτό φαίνεται άσχημο στο μάτι. Δεν φαίνεται να υπάρχει 
καμιά διακριτή διαφορά μεταξύ των διαφορετικών μεθόδων `greedy', `non-greedy' και
`post-disturbance'.

Μετά την αντιστοίχηση ιστογράμματος βλέπουμε πράγματι την αρχική εικόνα να 
έχει τιμές φωτεινότητας κοντινότερες σε αυτές της εικόνας αναφοράς. Όπως στην εικόνα
αναφοράς, το αποτέλεσμα φαίνεται να είναι πιο σκοτεινό γενικά. Υπάρχει ένα 
σημείο πάνω στη γάτα το οποίο είναι πολύ φωτεινό βέβαια και σε αυτό φαίνεται διαφορά 
στις τρεις μεθόδους. Η `post-disturbance' μέθοδος είναι πιο φωτεινή σε
εκείνο το σημείο. Στα ιστόγραμματα φαίνεται επίσης ότι η `post-disturbance'
μέθοδος έχει διαφορά από τις άλλες δύο στις χαμηλές φωτεινότητες.

\begin{figure}
    \centering
    \includegraphics[width=\textwidth]{Figure_1.png}
    \caption{Αρχική εικόνα, εικόνα αναφοράς και αντίστοιχα ιστογράμματα}
\end{figure}

\begin{figure}
    \centering
    \includegraphics[width=\textwidth]{Figure_2.png}
    \caption{Αποτελέσματα εξισορρόπησης ιστογράμματος στην αρχική εικόνα}
\end{figure}

\begin{figure}
    \centering
    \includegraphics[width=\textwidth]{Figure_3.png}
    \caption{Αποτελέσματα αντιστοίχησης ιστογράμματος αρχικής εικόνας με ιστόγραμμα
        εικόνας αναφοράς}
\end{figure}

\newpage

\vspace{2em}
\centering
\emph{***ΤΕΛΟΣ ΑΝΑΦΟΡΑΣ***}
\end{document}
