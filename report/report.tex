\documentclass{article} 
\usepackage{polyglossia} 
\usepackage{amsmath}
\usepackage{fontspec} 
\usepackage{lipsum} 
\usepackage[margin=1in]{geometry}
\usepackage{graphicx} 
\usepackage{caption} 
\usepackage{subcaption}
\usepackage{hyperref} 
\usepackage{verbatim}
\hypersetup{% 
    colorlinks=true, linkcolor=blue, filecolor=magenta,      
    urlcolor=cyan, 
    pdfinfo = {%
        Title = {ΨΕΕ 1η εργασία}
        Author = {Χρήστος Μάριος Περδίκης},
        Producer = XeLaTeX
    } 
}


\setmainfont{FreeSerif}


\title{Ψηφιακή Επεξεργασία Εικόνας \\ Παραλλαγές αλγορίθμων εξισορρόπησης και 
αντιστοίχισης ιστογράμματος}
\date{Εαρινό Εξάμηνο 2024-2025}
\author{Χρήστος-Μάριος Περδίκης 10075 cperdikis@ece.auth.gr}
 
\begin{document}
\maketitle

Αυτή είναι η αναφορά για την 1η εργασία του μαθήματος Ψηφιακής Επεξεργασίας 
Εικόνας. Ο σκοπός της εργασίας είναι η υλοποίηση μεθόδων εξισορρόπησης και 
αντιστοίχησης ιστογράμματος εικόνας σε python. Ακολουθεί η επεξήγηση των 
παραδοτέων συναρτήσεων.

\section{Συναρτήσεις στο αρχείο \texttt{hist\_utils.py}}
\subsection{\texttt{calculate\_hist\_of\_img()}}
Αρχικοποιείται το dictionary $hist$ με μήκος 256, ώστε να καλύψει όλες τις δυνατές 
τιμές μιας 8-bit εικόνας. Για κάθε pixel της εικόνας που έχει τιμή $f_i$, 
η αντίστοιχη τιμή του $hist$ αυξάνεται κατά ένα, δηλαδή $hist[f_i] = hist[f_i] + 1$.
Η συνάρτηση επιστρέφει το $hist$.

\subsection{\texttt{apply\_hist\_modification\_transform()}}
Αρχικοποιείται η νέα εικόνα $modified\_image$. Το κάθε pixel $(x, y)$ της λαμβάνει
τιμή $g_i$ που προέκυψε από την αντιστοιχία τιμών $g_i <-> f_i$ μέσω από το
Dict $modification\_transform$, όπου $f_i$ η τιμή του αντίστοιχου pixel $(x, y)$
στην εικόνα εισόδου $img_\array$. Η συνάρτηση επιστρέφει την εικόνα $modified\_image$.

\section{Συναρτήσεις στο αρχείο \texttt{hist\_modif.py}}
\subsection{\texttt{perform\_hist\_modification()}}
Ανάλογα με την τιμή της μεταβλητής $mode$, `greedy', `non-greedy', 
`post-disturbance', εκτελείται η αντίστοιχη μέθοδος αντιστοίχησης ιστογράμματος
μεταξύ του ιστογράμματος της εικόνας εισόδου $img\_array$ και του ιστογράμματος
αναφοράς $hist\_ref$. Αφού υπολογιστεί το ιστόγραμμα της εικόνας εισόδου 
με κλίση στη συνάρτηση \verb|calculate_hist_of_img|,
καλείται μία από τις βοηθητικές συναρτήσεις \verb|greedy_histogram_matching()|, 
\verb|non_greedy_histogram_matching()| και 
\verb|post_disturbance_histogram_matching()| ανάλογα με την τιμή της 
$mode$. Η αναλυτική επεξήγηση των συναρτήσεων αυτών βρίσκεται στις 
υποενότητες~\ref{helpfunc-greedy},~\ref{helpfunc-nongreedy} και~\ref{helpfunc-post}.
Για την ώρα, καλούνται με ορίσματα τα δύο ιστογράμματα, ένα της εικόνας εισόδου και ένα
αναφοράς, και επιστρέφουν την αντιστοίχηση μεταξύ των δύο ιστογραμμάτων σε 
μορφή Dict με όνομα $mod\_transform$.
\subsection{\texttt{perform\_hist\_eq()}}
\subsection{\texttt{perform\_hist\_matching()}}
\section{Αρχείο επίδειξης \texttt{demo.py}}
\section{Βοηθητικές συναρτήσεις}
\subsection{\texttt{greedy\_histogram\_matching()}}\label{helpfunc-greedy}
\subsection{\texttt{non\_greedy\_histogram\_matching()}}\label{helpfunc-nongreedy}
\subsection{\texttt{post\_disturbance\_histogram\_matching()}}\label{helpfunc-post}
\section{Αποτελέσματα συναρτήσεις}

\end{document}
