\documentclass{article} 
\usepackage{polyglossia} 
\usepackage{amsmath}
\usepackage{fontspec} 
\usepackage{lipsum} 
\usepackage[margin=1in]{geometry}
\usepackage{graphicx} 
\usepackage{caption} 
\usepackage{subcaption}
\usepackage{hyperref} 
\usepackage{verbatim}
\hypersetup{% 
    colorlinks=true, linkcolor=blue, filecolor=magenta,      
    urlcolor=cyan, 
    pdfinfo = {%
        Title = {Ημερολόγιο 1ης Εργασίας ΨΕΕ}
        Author = {Χρήστος Μάριος Περδίκης},
        Producer = XeLaTeX
    } 
}

\setmainfont{FreeMono}


\title{Ημερολόγιο 1ης Εργασίας ΨΕΕ \\ Ιστογράμματα}
\date{Εαρινό Εξάμηνο 2024-2025}
\author{Χρήστος-Μάριος Περδίκης 10075 cperdikis@ece.auth.gr}

\begin{document}
\maketitle

\section{23/4/25 --- Αρχή Των Πάντων}
Τρεις μέρες μόνο για εργασία πάμε λίγοοοοο!!! Διάβασα την εκφώνηση, ας σχεδιάσω 
τι σκοπεύω να κάνω γιατί το γιούρια της Γραφικής μου ήρθε μπούμερανγκ\ldots 

Στο ``hist\_utils.py'' υπάρχει η συνάρτηση:

\begin{verbatim}
    def calculate_hist_of_img(img_array: np.ndarray, return_normalized: bool):
        return hist: Dict
\end{verbatim}
Self-explanatory. Καλό σημείο να αρχίσω.

\begin{verbatim}
    def apply_hist_modification_transform(img_array: np.ndarray, mod_transform: Dict):
        return modified_image: np.ndarray
\end{verbatim}
Map τις στάθμες της εικόνας $img\_array$ στις στάθμες $g_i = mod\_transform
\left[f_i\right]$. Το $mod\_transform$ ποιός μας το δίνει όμως; Αυτές οι δύο 
συναρτήσεις θα καλούνται από τις ακόλουθες συναρτήσεις του αρχείου 
``hist\_modify.py''.

\begin{verbatim}
def perform_hist_modification(img_array: np.ndarray, hist_ref: Dict, mode: str):
    return modified_image: np.ndarray
\end{verbatim}
Conversion του ιστογράμματος της $img\_array$ σε αυτό του $hist\_ref$ ανάλογα 
με την τιμή του $mode$. Λογικά μέσα σε αυτή θα υπολογίζεται το $mod\_transform$
και θα καλείται η \verb|apply_hist_modification_transform|. Υπάρχει και η:

\begin{verbatim}
def perform_hist_eq(img_array: np.ndarray, mode: str):
    return equalized_img: np.ndarray
\end{verbatim}
Παραλλαγή της \verb|perform_hist_modification| απλά θα φτιάξω εγώ ένα uniform
ιστόγραμμα-αναφορά. Η συνάρτηση:

\begin{verbatim}
def perform_hist_matching(img_array: np.ndarray, img_array_ref: np.ndarray, mode: str):
    return processed_img: np.ndarray
\end{verbatim}
θα καλεί την \verb|perform_hist_modification| με την κατάλληλη προεργασία.

Στο αρχείο ``demo.py'' θα φορτώνονται οι εικόνες, θα καλούνται οι υπόλοιπες
συναρτήσεις και θα συγκρίνονται οι αρχικές με τις τελικές εικόνες και τα. Πρέπει 
να υπολογιστούν τρεις περιπτώσεις εξισορρόπησης ιστογράμματος για 
``greedy'', ``non-greedy'' και ``post-disturbance'' μεθόδους και άλλες τρεις
για αντστοίχηση ιστογράμματος.

Φαίνεται να κατέχω όλα τα κομμάτια του παζλ της εργασίας. Ας ξεκινήσω να την 
περιγράφω στον Claude.

\vspace{1em}
\hrule
\vspace{1em}

Ο Claude ᾽εκανε υπερβολικά καλή δουλειά\ldots{} Με την πρώτη ερώτηση έχω 
working example\ldots{} Νομίζω ας αρχίσω να κοιτάζω τι κώδικα έγραψε και να 
αρχίσω σιγά σιγά αναφορά --- κυρίως για να με βοηθήσει να καταλάβω τι σκατά έκανε 
ο Claude. Το μόνο λαθάκι που είχε είναι ότι έβαλε \verb|np.int| αντί για 
\verb|'int'| σε δύο casts \verb|np.astype()| στο ``demo.py''.
\end{document}
